\documentclass{../../../LaTeX/tdsimple}

\usepackage[francais]{babel}%
\usepackage[utf8]{inputenc}%
\usepackage{listings}%
\usepackage{graphicx}%
\usepackage{url}% 
\usepackage{hyperref}%

\newcommand{\daterendu}{27 octobre 2013}%
\newcommand{\datepresentation}{22 octobre 2013}%

\begin{document}

\title{%
  ENS Cachan Bretagne / Rennes\\
  L3 Informatique, parcours R \& I\\
  Cours de programmation, 1\ier\ semestre 2015--2016\\
  ---\\
  Projet~2\\
}


\author{Luc Bougé, Martin Quinson%
  \footnote{Rédigé d'après un document de Benoît Le Gouis, repris
    Pierre Karpman, repris d'un document de Matthieu Dorier, repris
    d'un document de Sébastien Chédor, repris d'un document de Thomas
    Gazagnaire.}}

\date{13 octobre 2015\\
  ---\\
  \lstinline[language={}]%
  $Id: projet2_Delaunay.tex 1595 2015-10-13 07:23:52Z bouge $%
}

% svn propset svn:keywords 'Id Revision' *.tex

\maketitle

\section*{Présentation du devoir}

Ce projet est à faire par groupes de~$2$ (plus un groupe de $1$ si
nécessaire).  La soutenance de ce projet aura lieu probablement le
\begin{center}
  \sl vendredi 6 novembre à \textbf{10h15}, sur le créneau du TD (à confirmer).
\end{center}
Les codes et les diapos de soutenance (format PDF) sont à envoyer à
\href{mailto:martin.quinson@ens-rennes.fr}{\email{martin.quinson@ens-rennes.fr}}
pour le
\begin{center}
  \sl mercredi 4~novembre à \textbf{7h00}.
\end{center}
Le rapport est à envoyer pour le
\begin{center}
  \sl dimanche 8~novembre à \textbf{18h00}.
\end{center}
Une pénalité d'un point par heure sera appliquée en cas de retard pour
le rendu du rapport.

\paragraph{Modalités de la soutenance.}
L'exposé de soutenance dure 10 minutes. Il est suivi de 10 minutes de questions.  Il
est souhaité que le groupe utilise un support visuel permettant de
rapidement résumer le travail effectué (comprenant typiquement des
captures d'écrans de résultats d'exécutions).  Les démonstrations de
programmes en direct sont aussi appréciées (quoique non obligatoires),
mais faites attention à bien gérer votre temps dans ce cas.

\paragraph{Contenu du rendu.}
Le rendu se fait sous la forme d'une archive contenant:
\begin{itemize}
\item Un rapport rédigé (en format PDF) expliquant et motivant les
  choix de conception (non triviaux) que vous avez fait, ainsi que les
  extensions que vous avez ajoutées au sujet de base.
\item Les diapos utilisés pour la soutenance.
\item Le code source de vos programmes (si possible, proposez deux
  versions pour chaque programme : une version de base répondant
  strictement aux questions et une version complète avec toutes vos
  extensions).
\item Un fichier \emph{README} donnant les commandes nécessaires à la
  compilation du (des) programme(s).
\end{itemize}
L'archive sera nommée d'après la convention \emph{Nom1Nom2}, les
noms de famille étant ordonnés alphabétiquement (par ex.
\emph{Calvin}, \emph{CalvinHobbes}, \emph{CalvinDerkinsHobbes}). Elle
sera envoyée par courrier électronique, l'objet duquel utilisera la
même convention que ci-dessus, préfixée de \emph{Prog1P1} (par
ex. \emph{Prog1P1CalvinHobbes}).

\begin{remarque}
  La qualité de la rédaction est très importante.  La notation prend
  en compte le soin, la structure et l'orthographe%
  \footnote{Pensez par exemple à utiliser un correcteur orthographique
    tel qu'\emph{aspell} si vous rédigez votre rapport avec \LaTeX.}
  du rapport.  L'aspect \emph{esthétique} de vos programmes est aussi
  important. Vous ferez donc attention à écrire un code commenté,
  lisible, bien indenté, au choix de vos noms de variables, etc. Il
  est recommandé d'utiliser un \emph{pretty-printer} Caml
  professionnel, par exemple \emph{ocp-indent} disponible avec \emph{OPAM}

  Par ailleurs, nous insistons pour que vous preniez l'habitude dès a
  présent de concevoir l'intégralité de votre code en anglais, que ce soit
  dans le nom des fichiers, le nom des variables et des procédure, les
  commentaires, etc. 
\end{remarque}

\subsection*{Évaluation}

Le sujet de projet tel que décrit dans ce document consiste en le
\emph{minimum nécessaire} pour obtenir une note passable.  Il vous est
fortement conseillé de réaliser des extensions aux programmes
demandés. Le sujet en donne quelques exemples, mais vous êtes libres
d'en choisir d'autres. Prenez cependant garde à ne pas être trop
ambitieux! Une extension modeste mais bien réalisée et fonctionnant
correctement sera plus valorisée qu'une qui serait plus complexe mais
implémentée de façon brouillonne et imparfaite. (Bien sûr, que ce
conseil ne soit pas non plus pris comme un encouragement à ne pas
faire de votre mieux!)

\subsection*{Assistance}

Martin Quinson et Luc Bougé sont à votre disposition pour répondre à
toutes vos questions. Notez que nous ne traitons que les questions
posées par l'interface Piazza
\url{https://piazza.com/ens-rennes.fr/fall2015/programmationl3ri/}.

\section{La triangulation de Delaunay}
Étant donné un ensemble de points du plan, on veut réaliser un
maillage de ce nuage de points par un ensemble de triangles les plus
réguliers possibles. Par réguliers, on entend\emph{ les plus
équilatéraux possibles}, ce qui se traduit par la propriété
suivante: l'intérieur du cercle circonscrit de chaque triangle ne
contient aucun point du nuages à mailler. Une triangulation qui
respecte ce critère est une triangulation de Delaunay. Un exemple de
triangulation de Delaunay est donné en Figure~\ref{fig:exdelaunay}.  La
Figure~\ref{fig:exs} présente une triangulation qui n'est pas correcte
(à gauche) et une qui l'est (à droite).

\begin{figure}
  \centering
  \includegraphics{Delaunay_triangulation.png}
  \caption{Exemple de triangulation de Delaunay}
  \label{fig:exdelaunay}
\end{figure}

\begin{figure}
  \centering
  \includegraphics[width=7cm]{circonscrit_1.png}
  \qquad
  \includegraphics[width=5cm]{circonscrit_2.png}
  \caption{Exemples de triangulations}
  \label{fig:exs}
\end{figure}

%%%%%%%%%%%%%%%%%%%%%%%%%%%%%%%%%%%%%%%

\section{Algorithme}

Il existe de nombreux algorithmes pour résoudre ce problème.  Celui
auquel nous allons nous consacrer est un algorithme incrémental non
optimal.  L'idée est la suivante: on génère dans un premier temps un
ensemble aléatoire de points, qui formeront le nuage à mailler.  On
initialise l'algorithme par un maillage trivial de la fenêtre
d'affichage (Figure~\ref{fig:maillageinit}), les points blancs sont
les points du nuage et les points noirs correspondent aux points
extrémaux visibles à l'écran. En d'autres termes, on ajoute les points
extrémaux à l'ensemble qu'on veut trianguler, et on les utilise comme
points de départ de la triangulation.

\begin{figure}
  \centering
  \includegraphics[width=8cm]{initial.png}
  \caption{Maillage initial}
  \label{fig:maillageinit}
\end{figure}

À chaque étape de l'algorithme on ajoute un nouveau point du nuage au
maillage.  Ce nouveau point peut appartenir aux cercles circonscrits à
certains triangles du maillage.  Il faut, d'une part, enlever ces
triangles du maillage (Figure~\ref{fig:ajoutpoint}, gauche) et,
d'autre part, y ajouter le maillage constitué du nouveau point et des
points des triangles enlevés (Figure~\ref{fig:ajoutpoint}, droite).

\begin{figure}
  \centering
  \includegraphics[width=5cm]{avant.png}
  \hfill
  \includegraphics[width=5cm]{apres.png}
  \caption{Ajout d'un point au maillage}
\label{fig:ajoutpoint}
\end{figure}

Il y a donc deux problèmes à résoudre:
\begin{enumerate}
\item à partir d'un ensemble de triangles et d'un point, trouver les
  triangles dont le cercle circonscrit contient le point;
\item à partir d'un ensemble de triangles à retirer et d'un point,
  calculer les triangles à ajouter.
\end{enumerate}

Pour résoudre le premier problème on remarquera que le point $D$ est à
l'intérieur du cercle circonscrit au triangle $(A,B,C)$ (donné dans le
sens direct) si et seulement si le déterminant de la matrice:
\[
\left(\begin{array}{cccc}
    A_x & A_y & A_x^2 + A_y^2 & 1\\
    B_x & B_y & B_x^2 + B_y^2 & 1\\
    C_x & C_y & C_x^2 + C_y^2 & 1\\
    D_x & D_y & D_x^2 + D_y^2 & 1
  \end{array}\right)
\]
est strictement positif.  Si $(A,B,C)$ est donné dans le sens indirect
le déterminant doit être strictement négatif.  Pour savoir si
$(A,B,C)$ est donné dans le sens direct il suffit de vérifier que le
déterminant de la matrice suivante est positif.
\[
\left(\begin{array}{cc}
    B_x - A_x & C_x - A_x \\
    B_y - A_y & C_y - A_y
  \end{array}\right)
\]
Pour résoudre le second problème on remarquera que les triangles à
retirer forment une zone convexe du plan.  On cherche à calculer la
frontière de cette zone.  Il suffit pour cela de remarquer qu'une
arête appartient à la frontière si et seulement si elle n'apparaît
qu'une seule fois dans l'ensemble des arêtes des triangles à retirer.
Pour ajouter le point $D$ il suffit donc, pour chacune de ces arêtes
$(A,B)$ d'ajouter un triangle $(A,B,D)$.

%%%%%%%%%%%%%%%%%%%%%%%%%%%%%%%%%%%%%%%%%%%%%%%%%%%%%%%%%%%%%%%%%%%%

\section{Structures de données et fonctions à implémenter}

Dans un premier temps on se limite aux structures permanentes: pas de
références, pas de champs mutables, etc.  Les types et les fonctions
proposés ici sont \emph{obligatoires} lors de votre première
implémentation de l'algorithme.

\subsection{Types}

\begin{lstlisting}
type point = {x: float; y: float}
type triangle = {p1: point; p2: point; p3: point}
type point_set = point list
type triangle_set = triangle list
\end{lstlisting}

\subsection{Fonctions}

\begin{itemize}

\item \|val random: int -> int -> int -> point list|:\\
  l'appel \|random nb max_x max_y| génère \|nb| points situés entre
  $(0,0)$ et $(\text{\|max_x|}, \text{\|max_y|})$.

\item \|val ccw: point -> point -> point -> bool|:\\
  l'appel \|ccw a b c| retourne \|true| si \|a|, \|b| et \|c| sont
  donnés dans le sens direct.

\item \|val in_circle: triangle -> point -> bool|:\\
  l'appel \|in_cricle triangle point| retourne \|true| si le point
  \|point| est à l'intérieur du cercle circonscrit au triangle
  \|triangle|.

\item \|val border: triangle_set -> (point * point) list|:\\
  \|border triangles| retourne une liste de couples $(p,q)$ tels que
  les points $p$ et $q$ sont deux points consécutifs de la frontière
  de la zone convexe définie par l'ensemble \|triangles|. Il faudra
  faire attention car les arêtes $(p,q)$ et $(q,p)$ sont identiques.
  
\item \|val add_point: triangle_set -> point -> triangle_set|:\\
  l'appel \|add_point triangles point| ajoute le point \|point| à la
  triangulation courante \|triangles| en utilisant l'algorithme
  proposé plus haut.

\item \|val delaunay: point_set -> int -> int -> triangle_set|:\\
  l'appel \|delaunay points max_x max_y| construit la triangulation de
  Delaunay du nuage de points \|points| en initialisant l'algorithme
  avec un maillage de l'espace de travail compris entre $(0,0)$ et
  $(\text{\|max_x|}, \text{\|max_y|})$.

\item \|val draw_points: point_set -> unit|:\\
  l'appel \|draw_points points| affiche à l'écran le nuage de points
  \|points|.

\item \|val draw_triangles: triangle_set -> unit|:\\
  l'appel \|draw_triangles triangles| affiche à l'écran la
  triangulation \|triangles|.

\item \|val delaunay_step_by_step: point_set -> unit|:\\
  l'appel \|delaunay_step_by_step points| construit en affichant
  chaque étape la triangulation de delaunay du nuage de points
  \|points|.

\end{itemize}

Pour la suite vous devrez toujours respecter ces signatures.  Les
seules choses qui peuvent être modifiées sont les types présentés plus
haut.  Le but est que les travaux de tous les groupes puissent être
combinés en un seul gros programme.

\section{Quelques extensions possibles}

Comme pour le premier projet, une extension ne sera corrigée et notée
que si tout ce qui précède est terminé de manière satisfaisante. Dans
l'idéal, cherchez à implémenter des extensions différentes pour chaque
groupe.
%Pour cette partie vous êtes assez libres: toute extension est
%recevable (parlez-en moi quand même avant, il ne s'agit pas de passer
%trop de temps sur quelque chose de trop compliqué, ou au contraire de
%faire quelque chose de trop simple).
Voici quelques suggestions d'extensions, mais ce n'est pas du tout
limitatif, bien au contraire, ô futurs chercheurs!

\paragraph{Triangulation sans points extrémaux.}

On peut souhaiter trianguler le nuage de point tel qu'il est généré,
sans avoir à inclure les points extrémaux de la fenêtre dans la
triangulation. Une façon de faire est de calculer l'enveloppe convexe
du nuage, puis d'exécuter le même algorithme que précédemment mais en
partant cette fois d'une triangulation de l'enveloppe convexe.

\paragraph{Ajout d'une coordonnée aux points.}

Ceci permet de tracer des approximations de courbes 3D en associant
une couleur à chaque triangle selon sa hauteur.  On peut aussi
travailler sur l'illumination d'un objet en 3D, la couleur d'un
triangle est alors fonction de son orientation par rapport à la source
lumineuse.  Enfin on peut afficher une vue 3D en gérant la position et
l'orientation d'une caméra (avec gestion intelligente de l'ordre
d'affichage des triangles).

\paragraph{Ajout d'un état aux triangles ou aux points.}

Ceci permet de construire un automate cellulaire sur une topologie un
peu tordue.

\paragraph{Triangulation dynamique.}

On a alors un ou plusieurs points mobiles.

\paragraph{Amélioration de la complexité.}

En utilisant des structures de données plus complexes (attention, les
fonctions doivent garder la signature proposée plus haut).  On tracera
des courbes pour évaluer expérimentalement la complexité et vérifier
si elle est vraiment meilleure que précédemment.

\end{document}

\documentclass[a4paper,11pt]{article}%
\usepackage{a4wide}%

\usepackage[francais]{babel}%
\usepackage[utf8]{inputenc}%
\usepackage[T1]{fontenc}%

\usepackage{graphicx}%
\usepackage{xspace}%

\usepackage{url} \urlstyle{sf}%
\DeclareUrlCommand\email{\urlstyle{sf}}%

\usepackage{mathpazo}%
\let\bfseriesaux=\bfseries%
\renewcommand{\bfseries}{\sffamily\bfseriesaux}

\newenvironment{keywords}%
{\description\item[Mots-clés.]}%
{\enddescription}


\newenvironment{remarque}%
{\description\item[Remarque.]\sl}%
{\enddescription}

\font\manual=manfnt
\newcommand{\dbend}{{\manual\char127}}

\newenvironment{attention}%
{\description\item[\dbend]\sl}%
{\enddescription}

\usepackage{listings}%

\lstset{%
  basicstyle=\sffamily,%
  columns=fullflexible,%
  language=caml,%
  frame=lb,%
  frameround=fftf,%
}%

\lstMakeShortInline{|}

\parskip=0.3\baselineskip%
\sloppy%

%%%%%%%%%%%%%%%%%%%%%%%%%%%%%%%%%%%%%%%%%%%%%%%%%

\begin{document}

\title{Le titre de votre projet}

\author{Vos noms}

\date{la date de rendu}

\maketitle

\begin{abstract}
  Quelques lignes de résumé, 5~au plus.
\begin{keywords}
 Quelques mots-clés, séparés par des points-virgules, au singulier,
 pour terminer le résumé.
\end{keywords}
\end{abstract}

\begin{attention}
  Un article \LaTeX\ est un programme. Son code \emph{source} doit être
  rédigé et mis en page avec autant de soin que votre code Caml,
  indépendamment de l'efficacité (l'élégance) du résultat de son
  exécution, c'est-à-dire le texte mis en page.
\end{attention}

\section{Pavage de Penrose}
\label{sec:Penrose}

Une première section.

Voici comment mettre du code Caml |let x = 1 in x+1;;| en
ligne. Attention, tout doit être sur la même ligne dans le fichier
source. Le caractère de séparation est défini par la commande
``lstMakeShortInline'' ci-dessus. Utilisez éventuellement des
commentaires \LaTeX\ pour garantir qu'il n'y a pas de coupure, comme ceci:
%
|let x = 1 in x+1;;|
%
comme ceci. 

Pour mettre du code en ``display'', utilisez l'environnement listings:
\begin{lstlisting}
let x = 1
in x+1;;
\end{lstlisting}
Tout est bien sûr paramétrable. Si vous avez du goût pour la mise en
page, n'hésitez pas!

Et voici comme insérer une figure:
\begin{figure}
  \centering
  \includegraphics[width=0.5\textwidth]{Obtus-divise}
  \caption{Voici une jolie figure. Attention, la commande ``label''
    doit être après la commande ``caption''.}
  \label{fig:example}
\end{figure}

\section{Tours de Hanoï}

Et une deuxième section.

\begin{remarque}
  Ici une remarque.
\end{remarque}

\begin{attention}
  Et ici un avertissement.
\end{attention}

\section{Conclusion}

Quelques mots de conclusion. Qu'avez-vous appris? Vous pouvez aussi insérer une bibliographie à la fin de votre rapport.

\clearpage % Pour être sûr que toutes les figures sont bien éjectées

\appendix % Passer en numérotation en lettres

\section{Annexe: auto-évaluation}

Toute l'explication de cette démarche se trouve à l'adresse
\url{http://fr.wikipedia.org/wiki/SWOT}. (Notez que ce lien est actif,
vous pouvez cliquer dessus.)
\begin{quotation}
  L'analyse (ou la matrice) SWOT est définie par les services de la
  Commission européenne comme un outil d'analyse stratégique. Il
  combine l'étude des forces et des faiblesses d'une organisation,
  d'un territoire, d'un secteur, etc. avec celle des opportunités et
  des menaces de son environnement, afin d'aider à la définition
  d'une stratégie de développement.
\end{quotation}

\subsubsection*{Points forts}
\begin{itemize}
\item Point 1.
\item Point 2.
\item etc.
\end{itemize}

\subsubsection*{Points faibles}
\begin{itemize}
\item Point 1.
\item Point 2.
\item etc.
\end{itemize}

\subsubsection*{Améliorations possibles}
\begin{itemize}
\item Point 1.
\item Point 2.
\item etc.
\end{itemize}

\subsubsection*{Difficultés à anticiper}
\begin{itemize}
\item Point 1.
\item Point 2.
\item etc.
\end{itemize}


\end{document}


\documentclass[a4paper,11pt]{article}%
\usepackage{a4wide}%

\usepackage[english]{babel}%
\usepackage[utf8]{inputenc}%
\usepackage[T1]{fontenc}%

\usepackage{graphicx}%
\usepackage{xspace}%

\usepackage{url} \urlstyle{sf}%
\DeclareUrlCommand\email{\urlstyle{sf}}%

\usepackage{mathpazo}%
\let\bfseriesaux=\bfseries%
\renewcommand{\bfseries}{\sffamily\bfseriesaux}

\newenvironment{keywords}%
{\description\item[Keyword.]}%
{\enddescription}


\newenvironment{remarque}%
{\description\item[Remark.]\sl}%
{\enddescription}

\font\manual=manfnt
\newcommand{\dbend}{{\manual\char127}}

\newenvironment{attention}%
{\description\item[\dbend]\sl}%
{\enddescription}

\usepackage{listings}%

\lstset{%
  basicstyle=\sffamily,%
  columns=fullflexible,%
  language=caml,%
  frame=lb,%
  frameround=fftf,%
}%

\lstMakeShortInline{|}

\parskip=0.3\baselineskip%
\sloppy%

%%%%%%%%%%%%%%%%%%%%%%%%%%%%%%%%%%%%%%%%%%%%%%%%%

\begin{document}

\title{Delaunay's triangulation \\ Project 2 PROG1}

\author{Simon Bihel}

\date{November 8, 2015}

\maketitle

\begin{abstract}
	This report will present and discuss my work on the second project of the course PROG1.
\begin{keywords}
	Delaunay; triangulation; convex hull.
\end{keywords}
\end{abstract}


\section{Basic program}

The obligatory part of the project was well guided and so I will talk about the approach I took to do the various functions. \\







\section{Advanced program}

The main extension that I decided to add was the triangulation without adding points, thus starting with the convex hull of the existing points. I will explain how I did it and then discuss the problem of nearly flat triangle that I have yet to solve.

\subsection{Convex hull}
To start the triangulation I had to compute which points were on the convex hull. I did it using the gift wrapping algorithm.

\subsection{Nearly flat triangle}


\section{Conclusion}


\clearpage % Pour être sûr que toutes les figures sont bien éjectées

\appendix % Passer en numérotation en lettres


\section{Annex: self evaluation}

\subsubsection*{Strengths}
\begin{itemize}
\item Straight forward approach, easy to understand.
\end{itemize}

\subsubsection*{Weaknesses}
\begin{itemize}
\item Non optimized ;
\item still existing bugs.
\end{itemize}

\subsubsection*{Opportunities}
\begin{itemize}
\item 
\end{itemize}

\subsubsection*{Threats}
\begin{itemize}
\item 
\end{itemize}


\end{document}

